\chapter{Cadre du stage}
	\section{Intégration continue et déploiement continue}
		\subsection{Intégration Continue}
			L'intégration continue (IC) est une pratique de développement logiciel qui vise à améliorer la qualité, la fiabilité et l'efficacité du processus de développement en automatisant la construction, les tests et la validation du code à chaque modification (les commits effectués par les développeurs, par exemple).\\
			Le principal but de cette pratique est de détecter les problèmes d'intégration au plus tôt lors du développement.\\
			L'intérêt de l'intégration continue réside dans plusieurs avantages clés :\\
			\begin{itemize}
				\item \textbf{Détection précoce des problèmes} : L'IC permet de détecter rapidement les erreurs et les problèmes de compatibilité en exécutant des tests automatisés à chaque fois qu'une modification de code est soumise. Cela aide à identifier les erreurs dès qu'elles sont introduites, facilitant ainsi leur correction avant qu'elles ne deviennent des problèmes plus graves.\\
				\item \textbf{Rapidité et réduction des retards} : En automatisant la compilation, les tests et les déploiements, l'IC accélère le processus de développement. Les développeurs n'ont pas besoin d'attendre de longues périodes avant de vérifier si leur code fonctionne correctement, ce qui réduit les retards et favorise une itération plus rapide.\\
				\item \textbf{Amélioration de la collaboration} : L'IC favorise la collaboration entre les membres de l'équipe en fournissant un environnement intégré où chaque développeur peut voir les modifications apportées par les autres membres de l'équipe et s'assurer qu'elles s'intègrent correctement. Cela réduit les conflits d'intégration et facilite le travail d'équipe.
			\end{itemize}
		
	\subsection{Déploiement continue}
		Le déploiement continu (DC) est une pratique de développement logiciel qui vise à automatiser et à simplifier le processus de mise en production des nouvelles fonctionnalités, des correctifs et des modifications de code. Contrairement à l'intégration continue qui se concentre sur l'automatisation des tests et de la validation à chaque modification de code, le déploiement continu va plus loin en automatisant également le déploiement des modifications validées dans un environnement de production.\\
		
		L'objectif principal du déploiement continu est de réduire le temps entre la validation d'une modification de code et sa disponibilité pour les utilisateurs finaux. Cela permet de livrer plus rapidement de nouvelles fonctionnalités, des améliorations et des correctifs, ce qui améliore l'expérience utilisateur et accroît l'agilité de l'équipe de développement.\\
	\section{Contexte}
	Pour automatiser 
	L'intégration et deploiement continue des codes produient par les développeurs de l'entreprise etaient effectuées par les FaaS (Function-as-a-Service) Jenkins et SQaaS (SonarQube).\\
	Jenkins