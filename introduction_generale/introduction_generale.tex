\chapter*{Introduction générale}

Depuis longtemps, La technologie n'a cessé de faire évoluer la productivité de milliers d'entreprises presque dans tous les secteurs d'activités.\\
Le passage de la mécanique aux domaines de l'informatique, de l'électronique de la domotique a révolutionné la vie journalière de l'être humain.
Aujourd'hui, vu l'intérêt croissant de vouloir gagner en temps, d'automatiser les tâches répétitives, cela a poussé petites, moyennes et grandes entreprises à chercher des solutions informatiques capables de répondre à leurs besoins.\\


Il y'a quelques années de cela, l'entreprise ONEMART disposait d'une application legacy qui permettait d'effectuer des \textbf{transferts} et \textbf{recouvrements} sur les différents numéros des ses clients qui sont appelés \textbf{intermèdes}. Cette application n'étant plus maintenable et ne pouvant pas interagir avec d'autre système externe ( principe d'\textbf{interopérabilité}), nous avons décider de refaire l'architecte de l'application en partant d'une conception simple.\\

C'est dans ce cadre s'inscrit notre projet de fin d'études qui consiste à réaliser une application de transfert et de recouvrement pour une entreprise appelée \textbf{ONEMART}.

Ainsi notre objectif a été de partir d'une application legacy et développer une nouvelle application qui pourra s'adapter dans le temps et pourra apporter un gain de temps de considérable à l'entreprise.\\

Ainsi pour bien cadrer l'étude de ce stage, notre travail se présentera comme suite :\\
\begin{itemize}
	\item En premier lieu nous allons présenter l'entreprise \textbf{ONEMART} et le cadre dans lequel se situera le projet.
	\item En second lieu nous allons faire la présentation des différentes phases de conception et l'étude technique du projet.
	\item Enfin nous présenterons l'application grâce aux différentes captures d'écran accompagnées de quelques détails.
\end{itemize}