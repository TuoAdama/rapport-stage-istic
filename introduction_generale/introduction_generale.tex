\chapter*{Introduction générale}
Dans le domaine du développement informatique, la montée en version d'une application est une chose qui intervient très souvent face aux besoins des clients qui ne cessent de s'accroître.\\
Pour s'adapter aux changements, le code d'une application doit être maintenable, testable et invulnérable face aux failles de sécurité. Pour le respect de ces contraintes, les outils de qualité de code ne peuvent qu'être indispensables dans le quotidien d'un développeur.\\
De tels outils, il en existe en nombre important selon la technologie utilisée (par exemple en PHP on peut citer: PHPMD, PHPCS, PHPSTAN, etc.). Ces outils permettent d'établir des règles de bonnes pratiques et générer des rapports sur l'état d'un projet.\\
Dans le cadre de ce stage, l'objectif est développer une application web qui permettra de récupérer tous ces rapports, les historiser dans afin de les visualiser de manière globale grâce à des diagrammes de différents types: lignes, camemberts, etc.