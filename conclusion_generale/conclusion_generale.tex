\chapter*{Conclusion générale}

	Dans le monde du développement informatique, pour mettre en place une application qui puisse être maintenable, il est important de suivre un chemin logique qui part de la conception jusqu'au choix des outils techniques pour la réalisation du projet.\\	C'est à partir de ce raisonnement que nous nous sommes posés de nombreuses questions quant au choix de la méthode de modélisation, de l'outil de développement, etc, pour mener à bien la réalisation de ce projet.
	Ce projet a été inspiré d'une application legacy (une vielle application) qui ne pouvait plus être maintenue par le fait de la non disponibilité de son code source et aussi de l'architecture sur laquelle elle reposait. C'est à partir de ce fait, que nous nous sommes fixés comme objectif de fournir une application qui bien évidemment basée sur une architecture simple pourra s'adapter aux besoins de l'entreprise et fournir plus de fonctionnalités dans le but de faciliter les différentes activités menées au sein de l'entreprise ONEMART.\\
	
	La réalisation de cette application nous a été bénéfique sur plusieurs points. Elle nous a permis de comprendre le sens réel de la pratique dans le domaine informatique car grâce à la pratique, nous avons touché à tout ce qui est API Rest, faire le tour du développement WEB et sans oublier la structuration de code que nous avons tirée de l'architecture MVC.\\
	
	En réalité, en partant de la conception jusqu'à la réalisation , le développement de cette application nous a également permis de nous perfectionner en améliorant nos connaissances en programmation informatique et en conception.Il aussi important de noter que grâce à ce projet, nous avons compris et mis en œuvre le déroulement d'un cycle de vie d'un logiciel.\\
	
	A l'issue de notre stage, nous d'autres idées d'amélioration nous sont venues en tête qui consiste à la mise en application d'une application mobile pour chaque agent l'entreprise.\\
	C'est ainsi que nous nous posés la question principale suivante :\\
	
	A l'aide des points d'entrées définies, comment mettre en place une application mobile pouvant effectuer les opérations de rechargements et transferts sans disposée de la puce e-master ?
	