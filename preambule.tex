\usepackage[english,french]{babel} %% Langue à utiliser dans le document
\usepackage[utf8]{inputenc} %% Encodage des caractères à saisir
%\usepackage{helvet}
\usepackage[T1]{fontenc} %% Encodage de l'affichage des caractères

%%Definition des marges de notre document
\usepackage[left=2cm,right=2cm,top=3cm,bottom=3cm]{geometry}

%%Package permettant d'utiliser les messages
\usepackage{graphicx}
    %\graphicspath{images/}

\usepackage{color}

%%Modification des entêtes de chapitre  
\usepackage[explicit]{titlesec}
\usepackage{tikz}
\usepackage{titletoc}
\usepackage{xpatch}
\definecolor{myblue}{RGB}{0,0,0}

\newcommand\DoPToC{%
    \startcontents[chapters]
    \printcontents[chapters]{}{1}{\noindent{\color{myblue}\rule{\textwidth}{1.5pt}}\par\medskip}%
}

\usepackage{fancyhdr}
 \setlength{\headheight}{9pt}
  \pagestyle{fancy}
  %\usepackage{lastpage}
  \renewcommand\headrulewidth{0.5pt}
  \fancyhead[L]{\scshape\leftmark\hspace{0,3cm}}
  \renewcommand\footrulewidth{0.5pt}
  \fancyfoot[C]{
      \textbf{
        \begin{tikzpicture}[baseline={([yshift=-.05ex]current bounding box.center)}]
          \node[fill=black,circle,text=white] {{{\thepage}
          }};
        \end{tikzpicture}
      }
  }
  \fancyhead[R]{2019-2020}

  \frenchbsetup{StandardLists=true} % à inclure si on utilise \usepackage[french]{babel}
  \usepackage{enumitem}
  \usepackage{amssymb}

	
  
  \usepackage{hyperref}
	\hypersetup{colorlinks,%
      citecolor=black,%
      filecolor=black,%
      linkcolor=black,%
      urlcolor=black
    }

\usepackage{tabularx}
\renewcommand{\tabularxcolumn}{m}    

%%Utilisation tikz pour des formes dans l'entête
\usepackage{tikz}
\usetikzlibrary{shadows.blur}
\usepackage{titletoc}


\usepackage[]{titlesec} 
\definecolor{yourcolor}{HTML}{000000}
\definecolor{noir}{HTML}{000000}

%%Couleur appliqué aux ecritures des chapitres
\colorlet{chpnumbercolor}{black}
\makeatletter
\let\oldl@chapter\l@chapter
\def\l@chapter#1#2{\oldl@chapter{#1}{\textcolor{chpnumbercolor}{#2}}}

\let\old@dottedcontentsline\@dottedtocline
\def\@dottedtocline#1#2#3#4#5{%
	\old@dottedcontentsline{#1}{#2}{#3}{#4}{{\textcolor{chpnumbercolor}{#5}}}}
\makeatother

\makeatletter
\xpatchcmd{\ttl@printlist}{\endgroup}{{\noindent\color{myblue}\rule{\textwidth}{1.5pt}}\vskip30pt\endgroup}{}{}
\makeatother



\titleformat%
{\chapter}[hang]%
{\bfseries}{%
	\begin{minipage}[t]{0.2\linewidth}  
		\vspace{0pt}% do not remove
		\begin{tikzpicture}
		\node[align=center,outer sep=0pt,text width=2.5cm,minimum height=2.5cm,fill=black,
		font=\color{white}\fontsize{80}{90}\selectfont,align=center
		] (num) {\thechapter};
		\node[rounded corners,outer sep=0pt,inner sep=0pt,anchor=south,
		font=\color{black}\Large\normalfont
		] at ([yshift=3pt]num.north) {{\textsc{}}};
		\end{tikzpicture}  
	\end{minipage}%
}
{0pt}%
{%
	\begin{minipage}[t]{.8\linewidth}%
		\vspace{1pt} % do not remove
		\rule{\linewidth}{2.5pt}\\\vskip -1.75\baselineskip%
		\rule{\linewidth}{.7pt}\vskip 5pt
		\begin{center}	
			\textsc{{\LARGE\raggedright\textsc{#1}}}
		\end{center}
	\end{minipage}%	
}

\titleformat{name=\chapter,numberless}[display]
{\normalfont\color{yourcolor}}
{
	{%
		\begin{minipage}[t]{1\linewidth}%
			\vspace{1pt} % do not remove
			\rule{\linewidth}{2.5pt}\\\vskip -1.75\baselineskip%
			\rule{\linewidth}{.7pt}\vskip 5pt
			\begin{center}	
				\textsc{{\LARGE\raggedright\textsc{#1}}}
			\end{center}
		\end{minipage}%	
	}
}
{1ex}
{\vspace*{.0001ex}\huge\sffamily\itshape}
[]

%command to print the acutal minitoc
\newcommand{\printmyminitoc}{%
	\noindent\hspace{-2cm}%
	\colorlet{chpnumbercolor}{white}%
	\begin{tikzpicture}
	\node[rounded corners,align=left,fill=yourcolor, blur shadow={shadow blur steps=5}, inner sep=5mm]{%
		\color{white}%
		\begin{minipage}{8cm}%minipage trick
		\printcontents[chapters]{}{1}{}
		\end{minipage}};
	\end{tikzpicture}}

\usepackage{nomencl} 
\makenomenclature 
\renewcommand{\nomname}{Liste des abréviations, des sigles}% Pour redéfinir le titre De cette liste
\input{front/sigles}

%\\usepackage[
	%\	backend=biber,        % compilateur par défaut pour biblatex
	%\	sorting=nyt,          % trier par nom, année, titre
	%\	citestyle=authoryear, % style de citation auteur-année
	%\	bibstyle=alphabetic,  % style de bibliographie alphabétique
	%\	]{biblatex}
%\addbibresource{mylit.bib}