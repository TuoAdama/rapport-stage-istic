\chapter{CADRE GÉNÉRAL}
\section{Introduction}
	Dans ce chapitre nous allons présenter ONEMART, la société pour laquelle nous avons effectué notre application de fin de cycle. Ensuite la présentation du cadre du projet permettra de mieux comprendre le problème étudié et présenter le principe de fonctionnement de l'application mise en place.
	Enfin, ce chapitre présentera la solution retenue qui sera détaillée plus loin dans ce mémoire.
\section{Présentation de l'entreprise}
	ONEMART est une société expérimentée dans la commercialisation des produits et services de la téléphonie mobile sur le marché national et bénéficiant d’un personnel hautement qualifié. Onemart dispose d’un réseau propre avec plusieurs points de vente mais aussi d’un portefeuille clientèle conséquent. Depuis janvier 2010, elle est le distributeur exclusif (franchisé) d’atlantique télécom CI, société de droit ivoirien propriétaire d’un réseau de radiotéléphonie cellulaire  exploité sous la marque Moov.\\
	
	ONEMART assure ainsi l’exclusivité de la distribution des produits et des services de Moov (Kits, recharges physiques et électroniques, portables et autres services après vente) dans les zones géographiques suivantes: Yopougon, Dabou, Sikensi, Tiassale, N’douci, Jacqueville, Grand-lahou, N’zianoua,.\\
	
	
\section{Cadre du projet} % Developpement web %
	Dans le cadre de notre mémoire nous avons concentrés notre énergie sur une des branches de l'entreprise qui est le rechargements des clients aussi appelés intermèdes.
\subsection{Problématique}
		Dans les années antérieures, l'entreprise ONEMART disposait d'une application VB déjà compilée avec toutes les plages de numéros dont elle disposait. Ainsi lors du passage des numérotations à dix chiffres, l'entreprise faisait face à un très gros problème qui était de mettre à jour les numéros des intermèdes vu que le code source de l'application n'était plus disponible.
		\paragraph{}A son siège, l'entreprise disposait d'un PC central câblé en local avec un téléphone d'ancienne génération(Sony Ericsson) équipé d'une puce \textbf{emaster} qui récupérait par câble les opérations effectuées (Recouvrement et transfert) sur le PC pour les valider. Ainsi, pour faire des rechargements et transferts depuis un point de vente distant, l'on devait se connecter par VPN (Virtual Private Network) pour avoir accès au PC central.En effet, puisque le VPN fait transiter la connexion de l'utilisateur par un serveur distant, ce qui rajoute une étape intermédiaire donc l'on fait face débit plus et lent moins stable.
		\paragraph{}Aussi sur le point sécuritaire, il est important de noter que les opérations se faisaient manuellement sachant bien que les montants de transferts était très élevés ce qui peut très problématique du moment où l'on peut être exposé aux problèmes suivants: Se tromper sur le montant de transfert, transférer l'argent d'un intermède	à un autre.\\
		
\subsection{Objectifs du projet}
	Face aux problèmes cités ci-dessus, notre application aura pour objectifs de :
	\begin{itemize}
		
		\item Mettre en place une application WEB car en plus d’être accessible sur toutes plateforme, elle et est simple et nécessite aucune installation.\\
		
		\item Que de passer par VPN (Virtual Private Network) pour avoir accès au PC central, l’application sera accessible par tous que ce soit en local ou sur internet.\\
		
		\item Après avoir effectuée les opérations et les stocker dans la base de données, l’application WEB pour les récupérer et les renvoyer à l’API (Application Programming Interface). En effet, l’interaction par API permettra au téléphone mobile de se passe par la connexion filaire pour avoir accès aux données et valider l'opération.
		
		\item 
		
	\end{itemize}
\subsection{Élaboration du cahier des charges}
	
	\begin{itemize}
		\item Créer un point d'entrer dans lequel les données seront accessible dans l'application.\\
		\item Persister les messages de transferts et recouvrement\\
		\item Sauvegarder les historique de transfert.\\
		\item Gérer la liste des intermèdes en ayant la possibilité de modifier, ajouter et supprimer.\\
		\item Gestion du mode de paiement (paiement par carte bancaire, crédit ou comptant).\\
		\item Élaboration de facture après chaque opération.\\
		\item Élaboration d'une facture après chaque opération avec possibilité de la télécharger.\\
		\item Gérer l'état des opérations:\\
			\begin{itemize}
				\item[$\bullet$]  état initié : Lorsque l'opération vient d'être effectuée.\\
				\item[$\bullet$] cours d'exécution: Lorsque l'application mobile récupère le message d'opération pour la valider.\\
				\item[$\bullet$] exécuté: Lorsque l'opération est effectuée par l'application.\\
				\item[$\bullet$] annulé: Lorsque le l'opération est annulée.
			\end{itemize}
	\end{itemize}

\subsection{Planification de l'application}
	Que l’on soit chef d’équipe ou responsable de sa propre activité, bien planifier ses projets est essentiel pour être efficace. Cela permet d’organiser son temps dépendamment du travail à réaliser, et de garantir son efficacité sur le long terme. Ainsi la planification d'une application passe par plusieurs étapes(qui peux dépendre d'une application à autre ) telle que:\\
	
	\begin{itemize}
		\item Étape n° 1 : \textbf{analyse fonctionnelle et définition des objectifs} : Cette partie consistera à rechercher et à caractériser les fonctions offertes par notre application pour satisfaire les différents besoins de l'agence.\\
		\item Étape n° 2 : \textbf{conception détaillée}: La phase de conception détaillée donne lieu à la rédaction du cahier des charges opérationnel. C'est elle qui précisera les différents éléments de dimensionnement du projet.\cite{planification} C'est aussi dans cette étape qu'entre la modélisation du système avec des méthodes de modélisations comme MERISE.\\
		\item Étape n° 3 : \textbf{développement du projet}: C'est dans partie qu'entre en jeu la partie technique de l'application. Cette étapes exige la maîtrise de la langage de programmation.\\
		\item Étape n° 4 : \textbf{Phase de tests}:
		%Tests unitaires, tests d'integration ,tests de validation%
		Cette phase un ensemble de test(tests unitaires, tests d'intégration et tests de validation) qui permettra de retrouver les erreurs moins évidentes qui n'ont pas été détectées pendant la phase de développement.\\
		\item Étape n° 5 : \textbf{recette}: Permet de s'assurer que l'application développée correspond bien aux exigences fixées par le client.\\
		\item Étape n° 6 : \textbf{mise en production}: Déploiement de l'application.\\
		\item Étape n° 7 : \textbf{maintenance}:On entend par maintenance, l'ensemble des modifications mises en place après la mise en oeuvre afin de corriger des bugs, améliorer les performances ou encore l'adapter à une modification de son environnement.\cite{planification}
	\end{itemize}
	
\section{Conclusion}
	Dans ce chapitre nous avons présenté ONEMART ainsi que ses différents activités.Nous avons cadré le projet sur lequel tient notre mémoire en définissant la problématique, les objectifs, le cahier des charges et enfin plan sur lequel se déroulera le développement de l'application.