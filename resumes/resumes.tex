\chapter*{Résumé}
	\thispagestyle{empty}
	
%	Faisant face à un système obsolète basé sur une architecture complexe, l'entreprise ONEMART retrouvait sa productivité altérée par le fait de la non-automatisation des tâches qu'elle effectuait.\\
%	C'est dans cette optique que s'est déroulé le stage de notre mémoire de Master. En effet tout au long de ce stage nous avons eu à mettre une application optimisée et maintenable répondant aux exigence de l'entreprise.\\
%	Afin carder le développement de notre application et la mise en place d'une architecture solide, nous avons fait appel à des méthodes de conceptions impeccables dans le but de mener à bien ce projet.
	
	Faisant face à un système obsolète basé sur une architecture complexe, l'entreprise ONEMART retrouvait sa productivité altérée par le fait de la non-automatisation des tâches qu'elle effectuait.\\
	C'est dans cette optique que s'est déroulée le stage de notre mémoire de Master. En effet tout au long de ce stage nous avons eu à mettre une application Web de transferts et recouvrement d'argent à une plage de client ( aussi appelé intermède) suivant un mode de paiement donnée (Crédit, comptant, bancaire, etc ...). Une fois l'opération (transfert ou recouvrement) effectuée, grâce à l'interopérabilité offerte par l'application, un système externe (nous avons aussi appelé validateur) récupère les informations de l'opération et la valide grâce à une syntaxe USSD.\\
	Ainsi Afin de cadrer le développement de notre application et de répondre aux exigences de l'entreprise, nous avons élaboré un plan de conception grâce aux différents diagrammes UML. Vu les fonctionnalités offertes par l'application, elle a été développée en utilisant le framework \textbf{Laravel} et d'autres outils de développement (le système de gestion de base de donnée relationnelle \textbf{MySQL}, une bibliothèque JavaScript \textbf{AlpineJs}, etc...) pour faciliter le développement de l'application tout en assurant une bonne sécurité de celle-ci.\\
	
	\textbf{Mots clés}: Interopérabilité, Intermèdes, Transferts, Recouvrements, Rechargements
	
%	Faire des transfert à un plage d'intermède,
%	Intermède fait le transfert suivant un mode de paiement
%	A l'aide de quels outils ?
%	Comment avons nous 	réglé ce problème ?
%	