\chapter{REALISATION DE L’APPLICATION}
	\section{Introduction}
		Dans ce chapitre, nous présenterons les outils de développement et par la suite ainsi que les capture de l’application.
	\section{Outils de développement}
		\subsubsection{Technologie front-end }
		\subsubsection{HTML 5}
		 L'HyperText Markup Language, HTML, désigne un type de langage informatique descriptif. Il s'agit plus précisément d'un format de données utilisé dans l'univers d'Internet pour la mise en forme des pages Web. Il permet, entre autres, d'écrire de l'hypertexte, mais aussi d'introduire des ressources multimédias dans un contenu.\\
		 
		 Développé par le W3C (World Wide Web Consortium) et le WHATWG (Web Hypertext Application Technology Working Group), le format ou langage HTML est apparu dans les années 1990. Il a progressivement subi des modifications et propose depuis 2014 une version HTML5 plus aboutie.
		Dans cette page on peut voir la liste des agents autorisés à accéder à l'application.\\
		
		L'HTML est ce qui permet à un créateur de sites Web de gérer la manière dont le contenu de ses pages Web va s'a‑cher sur un écran, via le navigateur. [4]\\
		
		
		
	\section{Ajouter d'un agent}
		Page permettant ajouter des agents tout en ayant la possibilité de leurs attribuer des rôles.
		\begin{center}
			\includegraphics[scale=0.4]{chap_3/ajouter_agent.png}
			\captionof{figure}{Ajouter un agent}
			\label{ajouter_agents}
		\end{center}
	\section{Les Rôles}
		Pour définir une restriction à l'accessibilité aux ressources de l'application, on attribue un ou plusieurs rôle à un utilisateur.
		\begin{center}
			\includegraphics[scale=0.4]{chap_3/roles.png}
			\captionof{figure}{Les rôles}
			\label{les_roles}
		\end{center}
	\section{Ajouter intermède}
		\subsection*{État civil}
		\begin{center}
			\includegraphics[scale=0.4]{chap_3/ajouter-intermed-etat-civil.png}
			\captionof{figure}{Ajouter intermède}
			\label{ajouter_intermede-etat-civil}
		\end{center}
		\subsection*{Les numéros}
			\begin{center}
				\includegraphics[scale=0.4]{chap_3/ajout-intermed-numeros.png}
				\captionof{figure}{Ajout de numéros}
				\label{ajouter_intermede-les-numeros}
			\end{center}
		
		\subsection*{Photo de profil de l'intermède}
		\begin{center}
			\includegraphics[scale=0.4]{chap_3/ajout-intermed-dossier.png}
			\captionof{figure}{}
			\label{ajouter_intermede-dossier}
		\end{center}
			
	\section{Liste des intermèdes}
		Dans cette on affiche tous les intermèdes avec leur solde.\\
		\begin{center}
			\includegraphics[scale=0.41]{chap_3/liste_intermède.png}
			\captionof{figure}{Liste des intermèdes}
			\label{liste_intermede}
		\end{center}
	\section{Les modes de paiements}
		Cette section concerne les modes de paiements. Nous avons mis en place un système qui pourra s'adapter dans le temps  car elle permettra d'ajouter si on le veut d'autre mode de paiement.\\
		\begin{center}
			\includegraphics[scale=0.4]{chap_3/mode_paiement.png}
			\captionof{figure}{Mode de paiement}
			\label{mode_de_paiement}
		\end{center}
	\section{Liste des types}
		Dans cette partie également on liste l'ensemble d'opérations que peut effectuer l'agence pour que plus tard on puisse ajouter d'autre nouvelle fonctionnalité.
		\begin{center}
			\includegraphics[scale=0.4]{chap_3/types.png}
			\captionof{figure}{Les types}
			\label{mode_de_paiment}
		\end{center}
	\section{Recouvrement}
		Le recouvrement ou encore dépôt se sert des informations présentes sauf que le mode de paiement à crédit n'est pas accessible.\\
		\begin{center}
			\includegraphics[scale=0.4]{chap_3/recouvrement_1.png}
			\captionof{figure}{Recouvrement}
			\label{recouvrement_1}
		\end{center}
		\begin{center}
			\includegraphics[scale=0.4]{chap_3/recouvrement_2.png}
			\captionof{figure}{Details Recouvrement}
			\label{recouvrement_2}
		\end{center}
		\begin{center}
			\includegraphics[scale=0.4]{chap_3/recouvrement_3.png}
			\captionof{figure}{Confirmation recouvrement}
			\label{recouvrement_3}
		\end{center}
	\section{Transfert}
		Bien que l'interface de transfert soit similaire à celle du recouvrement, dans le l'opération de transfert on a accès à tous les modes de paiements.
		\begin{center}
			\includegraphics[scale=0.4]{chap_3/transfert_1.png}
			\captionof{figure}{Transfert}
			\label{transfert_1}
		\end{center}
		\begin{center}
			\includegraphics[scale=0.4]{chap_3/transfert_2.png}
			\captionof{figure}{Detail du transfert}
			\label{transfert_2}
		\end{center}
	\section{Facture après une opération}
		Après chaque opération, une facture est délivré à l'intermède. Cette facture se présente comme suite :
		\begin{center}
			\includegraphics[scale=0.7]{chap_3/facture.png}
			\captionof{figure}{Présentation de facture}
			\label{Facture}
		\end{center}
	\section{Message des opérations}
		Une fois qu'une opération est effectuée, on a un message qui est généré.
		\begin{center}
			\includegraphics[scale=0.5]{chap_3/message_1.png}
			\captionof{figure}{Message des opérations}
			\label{message}
		\end{center}
		\begin{center}
			\includegraphics[scale=0.5]{chap_3/message_2.png}
			\captionof{figure}{Message des transferts}
			\label{message_transfert}
		\end{center}
	\section{États des paiements}
		\begin{center}
			\includegraphics[scale=0.5]{chap_3/etat_de_paiement.png}
			\captionof{figure}{Etat de paiement}
			\label{etat_de_paiement}
		\end{center}
	\section{Conclusion}
		Dans ce chapitre nous avons exploré les grandes parties de notre application grâce aux différentes captures d'écrans et quelque détails pour plus de compréhension.